\subsection{Breen--Deligne data}

The goal of this subsection is to a give a precise statement of
a variant of the Breen--Deligne resolution.
This variant is not actually a resolution,
but it is sufficient for our purposes,
and is much easier to state and prove.

We first recall the original statement of the Breen--Deligne resolution.
\begin{theoremx}[Breen--Deligne]
  \label{BD_orig}
  For an abelian group $A$, there is a resolution, functorial in~$A$, of the form
  \[
    \ldots \to \bigoplus_{i=1}^{n_i} \mathbb Z[A^{r_{ij}}] \to \ldots
    \to \mathbb Z[A^3] \oplus \mathbb Z[A^2] \to \mathbb Z[A^2] \to \mathbb Z[A] \to A \to 0.
  \]
\end{theoremx}

What does a homomorphism $f \colon \mathbb Z[A^m] \to \mathbb Z[A^n]$
that is functorial in~$A$ look like?
We should perhaps say more precisely what we mean by this.
The idea is that $m$ and $n$ are fixed,
and for each abelian group $A$ we have a group homomorphism
$f_A\colon \mathbb Z[A^m] \to \mathbb Z[A^n]$ 
such that if $\phi \colon A\to B$ is a group homomorphism
inducing $\phi_i \colon\Z[A^i]\to\Z[B^i]$ for each natural number $i$
then the obvious square commutes: $\phi_n \circ f_A = f_B \circ \phi_m$.

The map $f_A$ is specified by what it does to the generators
$(a_1, a_2, a_3, \dots, a_m)\in A^m$.
It can send such an element to an arbitrary element of $\mathbb Z[A^n]$,
but one can check that universality implies that $f_A$
will be a $\mathbb Z$-linear combination of ``basic universal maps'',
where a ``basic universal map'' is one that sends $(a_1, a_2, \dots, a_m)$ to $(t_1, \dots, t_n)$,
where $t_i$ is a $\mathbb Z$-linear combination $c_{i,1} \cdot a_1 + \dots + c_{i,m} \cdot a_m$.
So a ``basic universal map'' is specified by the $n \times m$-matrix $c$.

\begin{definition}
  \label{basic_universal_map}
  \lean{breen_deligne.basic_universal_map}
  \leanok
  A \emph{basic universal map} from exponent $m$ to $n$,
  is an $n \times m$-matrix with coefficients in~$\mathbb Z$.
\end{definition}

\begin{definition}
  \label{universal_map}
  \lean{breen_deligne.universal_map}
  \leanok
  \uses{basic_universal_map}
  A \emph{universal map} from exponent $m$ to $n$,
  is a formal $\mathbb Z$-linear combination
  of basic universal maps from exponent $m$ to $n$.

  If $f$ is a basic universal map,
  then we write $[f]$ for the corresponding universal map.
\end{definition}

\begin{definition}
  \label{universal_map_bound_by}
  \lean{breen_deligne.universal_map.bound_by}
  \leanok
  \uses{universal_map}
  Let $f = \sum_g n_g[g]$ be a universal map.
  We say that $f$ is \emph{bound by} a natural number $N$
  if $\sum_g |n_g| \le N$.
\end{definition}

We point out that basic universal maps can be composed by matrix multiplication,
and this formally induces a composition of universal maps.
As mentioned above, one can also check (this has been formalised in Lean)
that this construction gives a bijection
between universal maps from exponent $m$ to $n$
and functorial collections $f_A \colon \Z[A^m]\to\Z[A^n]$.

\begin{definition}
  \label{FreeMat}
  \uses{universal_map}
  \lean{breen_deligne.FreeMat}
  \leanok
  In other words, we are considering the following two categories:
  \begin{itemize}
    \item
      the category whose objects are natural numbers,
      and whose morphisms are matrices;
    \item
      the category with the same objects,
      but with Hom-sets replaced by the free abelian groups
      generated by the sets of matrices.
      We denote this latter category $\FreeMat$.
  \end{itemize}
\end{definition}

Both categories naturally come with a monoidal structure:
for the first it is given by the Kronecker product of matrices
(a.k.a. tensor product of linear maps)
which induces a monoidal structure on $\FreeMat$.
As usual, we will denote this monoidal structure $\_ \otimes \_$.
For example, if $f$ is a basic universal map,
then $2 \otimes f$ denotes the block matrix
\[
  \begin{pmatrix}
    f & 0 \\
    0 & f
  \end{pmatrix}
\]

\begin{definition}
  \label{proj_aux}
  \uses{basic_universal_map}
  \lean{breen_deligne.basic_universal_map.proj_aux}
  \leanok
  Let $N$ be a natural number, and $i < N$.
  Then $\pi'_{N,i}$ denotes the basic universal map
  from exponent $N$ to $1$
  \[
   \begin{pmatrix} 
    0 \\
    0 \\
    \vdots \\
    1 \\
    \vdots \\
    0
   \end{pmatrix} 
   =
   (a_0, a_1, \ldots, a_{N-1})^t
  \]
  where $a_j = \delta_{ij}$.
\end{definition}

\begin{definition}
  \label{proj}
  \uses{proj_aux}
  \lean{breen_deligne.universal_map.proj}
  \leanok
  Let $N$ and~$n$ be natural numbers.
  Then $\pi^N_n$ denotes the universal map from exponent $N \cdot n$ to $n$
  given by $\sum_{i < N} [\pi'_{N,i} \otimes n]$.

  (On $\Z[A^{N \cdot n}] \to \Z[A^n]$ this map is
  the formal sum of the maps $\Z[A^{N \cdot n}] \to \Z[A^n]$
  induced by the projection maps $A^{N \cdot n} = (A^n)^N \to A^n$.)
\end{definition}

\begin{definition}
  \label{sum}
  \uses{proj_aux}
  \lean{breen_deligne.universal_map.sum}
  \leanok
  Let $N$ and~$n$ be natural numbers.
  Then $\sigma^N_n$ denotes the universal map from exponent $N \cdot n$ to $n$
  given by $[\sum_{i < N} \pi'_{N,i} \otimes n]$.

  (On $\Z[A^{N \cdot n}] \to \Z[A^n]$ this map is
  induced by the summation map $A^{N \cdot n} = (A^n)^N \to A^n$.)
\end{definition}

\begin{definition}
  \label{BD_data}
  \lean{breen_deligne.data}
  \leanok
  \uses{FreeMat}
  A \emph{Breen--Deligne data} is a chain complex in $\FreeMat$.

  Concretely, this means that it
  consists of a sequence of exponents $n_0, n_1, n_2, \dots \in \mathbb N$,
  and universal maps $f_i$ from exponent $n_{i+1}$ to $n_i$,
  such that for all $i$ we have $f_i \circ f_{i+1} = 0$.

  A morphism of Breen--Deligne data is a morphism of chain complexes.
\end{definition}

\begin{definition}
  \label{BD_mul}
  \lean{breen_deligne.data.mul}
  \leanok
  \uses{BD_data}
  For every natural numbers $N$, the endofunctor $N \otimes \_$ on $\FreeMat$
  induces an endofunctor of Breen--Deligne data.

  Concretely, it maps a pair $(n, f)$ of Breen--Deligne data,
  to the pair $N \otimes (n,f)$ consisting of exponents $N \cdot n_i$
  and universal maps $N \otimes f_i$.
\end{definition}

Let $\BD$ be Breen--Deligne data.
The universal maps $\sigma^N$ and~$\pi^N$ defined above,
induce morphisms $\sigma^N_\BD,\pi^N_\BD \colon N \otimes \BD \to \BD$.

\begin{definition}
  \label{BD_package}
  \lean{breen_deligne.package}
  \leanok
  \uses{BD_data, BD_mul, proj, sum}
  A \emph{Breen--Deligne} package consists of Breen--Deligne data $\BD$
  together with a homotopy $h$ between $\pi^2_\BD$ and $\sigma^2_\BD$. 
\end{definition}

\begin{definition}
  \label{BD_h_mul}
  \uses{BD_package}
  \lean{breen_deligne.data.homotopy_mul}
  \leanok
  Let $\BD$ be a Breen--Deligne package and $N$ a power of $2$.
  Then the homotopy $h$ induces a homotopy
  between $\pi^N_\BD$ and $\sigma^N_\BD$ by iterative composition
  of the homotopy packaged in $\BD$.
\end{definition}

\begin{definition}
  \label{BD_eg}
  \lean{breen_deligne.eg}
  \leanok
  \uses{BD_package}
  We will now construct an example of a Breen--Deligne package.
  In some sense, it is the ``easiest'' solution to the conditions posed above.
  The exponents will be $n_i = 2^i$, and the homotopies $h_i$ will be the identity.
  Under these constraints, we recursively construct the universal maps $f_i$:
  \[
    f_0 = \pi^2_1 - \sigma^2_1,
    \quad
    f_{i+1} = (\pi^2_{2^{i+1}} - \sigma^2_{2^{i+1}}) - (2 \otimes f_i).
  \]
  We leave it as exercise for the reader, to verify that
  with these definitions $(n, f, h)$ forms a Breen--Deligne package.
\end{definition}

We now make definitions that will make precise
some conditions between constants that will be needed
when we construct Breen--Deligne complexes of normed abelian groups.

\begin{definition}
  \label{basic_suitable}
  \lean{breen_deligne.basic_universal_map.suitable}
  \leanok
  \uses{basic_universal_map}
  Let $f$ be a basic universal map from exponent~$m$ to~$n$.
  Let $c_1, c_2 \in \mathbb R_{\ge 0}$.
  We say that $(c_1, c_2)$ is \emph{$f$-suitable}, if for all $i$
  \[
    \sum_j c_1|f_{ij}| \le c_2.
  \]
\end{definition}

To orient the reader:
later on we will be considering maps on normed abelian groups induced from universal maps,
and this inequality will guarantee that if $\|m\|\leq c_1$ then $\|f(m)\|\leq c_2$.

\begin{definition}
  \label{universal_suitable}
  \lean{breen_deligne.universal_map.suitable}
  \leanok
  \uses{universal_map, basic_suitable}
  Let $f$ be a universal map from exponent~$m$ to~$n$.
  Let $c_1, c_2 \in \mathbb R_{\ge 0}$.
  We say that $(c_1, c_2)$ is \emph{$f$-suitable},
  if for all basic universal maps $g$
  that occur in the formal sum $f$,
  the pair of nonnegative reals $(c_1, c_2)$ is $g$-suitable.
\end{definition}

\begin{definition}
  \label{very_suitable}
  \lean{breen_deligne.universal_map.very_suitable}
  \leanok
  \uses{universal_suitable, universal_map_bound_by}
  Let $f$ be a universal map and let $r, r', c_1, c_2 \in \mathbb R_{\ge 0}$.
  We say that $(c_1, c_2)$ is \emph{very suitable}
  for $(f, r, r')$
  if there exist $N, b \in \N$ and $c' \in \R_{\ge 0}$ such that:
  \begin{itemize}
    \item $f$ is bound by $N$ (see Definition~\ref{universal_map_bound_by})
    \item $(c_1, c')$ is $f$-suitable
    \item $r ^ b N ≤ 1$
    \item $c' ≤ (r') ^ b c_2$
  \end{itemize}
\end{definition}

\begin{definition}
  \label{BD_suitable}
  \label{BD_very_suitable}
  \lean{breen_deligne.data.suitable, breen_deligne.data.very_suitable}
  \leanok
  \uses{BD_data, very_suitable}
  Let $\BD = (n, f)$ be Breen--Deligne data,
  let $r, r' \in \R_{\ge 0}$,
  and let $\kappa = (\kappa_0, \kappa_1, \dots)$ be a sequence of nonnegative real numbers.
  We say that $\kappa$ is $\BD$-\emph{suitable}
  (resp.\ \emph{very suitable} for $(\BD, r, r')$),
  if for all $i$, the pair $(\kappa_{i+1}, \kappa_i)$ is $f_i$-suitable
  (resp.\ \emph{very suitable} for $(f_i, r, r')$).

  (Note! The order $(\kappa_{i+1}, \kappa_i)$ is contravariant
  compared to Definition~\ref{universal_suitable}.
  This is because of the contravariance of $\hat V(\_)$;
  see Definition~\ref{eval_CLCFPTinv}.)
\end{definition}

\begin{definition}
  \label{adept}
  \lean{breen_deligne.package.adept}
  \leanok
  \uses{BD_package, universal_suitable}
  Let $\BD$ be a Breen--Deligne package with data $(n,f)$ and homotopy $h$.
  Let $\kappa, \kappa'$ be sequences of nonnegative real numbers.
  (In applications $\kappa$ is a $(n,f)$-suitable sequence.)

  Then $\kappa'$ is \emph{adept} to $(\BD, \kappa)$ if
  for all $i$ the pair $(\kappa_i / 2, \kappa'_{i+1} \kappa_{i+1})$
  is $h_i$-suitable.
  (Recall that $h_i$ is the homotopy map $n_i \to n_{i+1}$.)
\end{definition}

\begin{lemma}
  \label{BD_h_mul_suitable}
  \lean{breen_deligne.package.adept.homotopy_mul_suitable}
  \leanok
  \uses{adept, BD_h_mul}
  Let $\BD$ be a Breen--Deligne package, $N$ a power of $2$, and
  let $\kappa, \kappa'$ be sequences of nonnegative real numbers.
  Assume that $\kappa'$ is adept to $(\BD, \kappa)$.
  Let $h^N$ be the homotopy between $\pi^N_\BD$ and $\sigma^N_\BD$
  defined in Def~\ref{BD_h_mul}.

  For all $i$, the pair $(\kappa_i / N, \kappa'_{i+1} \kappa_{i+1})$
  is $h^N_i$-suitable.
\end{lemma}

\begin{proof}
  \leanok
  Omitted. (But done in Lean.)
\end{proof}

\begin{lemma}
  \label{exists_very_suitable}
  \lean{breen_deligne.data.c_very_suitable}
  \leanok
  \uses{BD_very_suitable}
  Let $\BD$ be a Breen--Deligne package, and
  let $r, r'$ be nonnegative reals, such that $r < 1$ and $r' > 0$.

  There exists a sequence $\kappa$ of positive real numbers
  such that $\kappa$ is very suitable for $(\BD, r, r')$.
\end{lemma}

\begin{proof}
  \leanok
  The sequence can be constructed recursively,
  which we leave as exercise for the reader.
  (It has been done in Lean.)
\end{proof}

\begin{lemma}
  \label{exists_adept}
  \lean{breen_deligne.package.κ'_adept}
  \leanok
  \uses{adept}
  Let $\BD$ be a Breen--Deligne package, and
  let $r, r'$ be nonnegative reals, such that $0 < r < 1$ and $0 < r' \le 1$.
  Let $\kappa$ be any sequence of positive reals.

  There exists a sequence $\kappa'$ of nonnegative real numbers that is adept to $(\BD, \kappa)$.
\end{lemma}

\begin{proof}
  \leanok
  The sequence can be constructed recursively,
  which we leave as exercise for the reader.
  (It has been done in Lean.)
\end{proof}

% vim: ts=2 et sw=2 sts=2

