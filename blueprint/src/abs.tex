adfe develop some 
homological and element-wise machinery
from Commutative Algebra, culminating 
in the proof of the Auslander-Buchsbaum-Serre 
theorem, and 
the corollary that the condition of being 
a regular ring localizes. 
This corollary is crucial for our proof of adjunction.

We follow Bruns-Herzog almost exclusively.

The current version of this blueprint is incomplete--this
reflects the fact that we have not yet chosen which proof
of Auslander-Buchsbaum-Serre to formalize.

In particular, there are two detailed by Bruns-Herzog:
one uses a fact about Cohen-Macaulay rings for the forward
direction, and a theorem of Ferrand-Vasconcelos for the
reverse direction. 
The other uses Koszul homology and some facts about Tor
for the forward direction, and a statement of Serre 
on DG-algebras for the reverse direction.
We can basically pick and choose which proof we want to use
for each direction.

%TODO: Bring all Nakayama corollaries 
%      here into section one, prove them,
%      and then add references to the correct
%      corollaries anywhere that 'nakayama' or 'Nakayama'
%      appears in this document.


\subsection{Nakayama's Lemma and Corollaries}

\begin{lemma}[Nakayama's Lemma]
  \label{lem:nakayama}
  State Nakayama's Lemma here.
\end{lemma}

\begin{proof}
  The proof of this is already in Mathlib
\end{proof}

\begin{corollary}
  \label{cor:local_maximal_gereating_set_basis_residue}
  \uses{lem:nakayama}
  Let $(R,\mathfrak{m},k)$ be a local ring. 
  Let $M$ be an $R$-module.
  If the elements $x_1, \ldots, x_n$ are elements in
  $M$ that form a basis in the projection 
  $M / \mathfrak{m} M$, then $x_1, ldots, x_n$ generate $M$.
\end{corollary}

\begin{proof}
  See Atiyah-MacDonald Corollary 2.8
\end{proof}

The following lemma is used by user6:

\begin{lemma} \label{lem:fin_gen_proj_over_local_is_free}
  \uses{cor:local_maximal_gereating_set_basis_residue}
	A finitely generated projective module 
	over a regular local ring is free
\end{lemma}

\begin{proof}
	A proof of this theorem can be pieced together from this
	stack exchange answer:
	https://math.stackexchange.com/questions/3362463/projective-modules-over-local-rings-are-free-matsumuras-proof
	This proof needs
	\begin{itemize}
		\item Nakayama's lemma
		\item Equivalent definitions of projective modules
	\end{itemize}
\end{proof}

\begin{lemma}
  \label{lem:surj_endo_is_isom}
  \uses{cor:local_maximal_gereating_set_basis_residue}
  Let $f : M \to M$ be a surjection of modules 
  (over a local ring?).
  Then $f$ is an isomorphism.
\end{lemma}

\begin{proof}

  I think this uses the Corollary 2.8 version of Nakayama's Lemma
\end{proof}


\subsection{Projective Resolutions}

\begin{definition}
  \label{def:projv_resl}
  Let $R$ be a ring. 
  Let $M$ be an $R$-module.
  A projective resolution of $M$ over $R$
  Is an exact sequence
  \[
    \ldots \to P^2 \to P^1 \to P^0 \to M \to 0
  \].
  That is to say, a projective resolution is a
  quasi-isomorphism of complexes between the 
  inclusion of $M$ into chain complexes over $R$
  (i.e. the complex which has $M$ in degree zero,
  trivial everywhere else, with trivial maps)
  and a bounded—below complex whose
  components are projective $R$-modules.
\end{definition}

\begin{definition}
  \label{def:free_resl}
  \uses{def:projv_resl}
  A free resolution of $M$ over $R$ is 
  a projective resolution whose components
  $P^i$ are free.
\end{definition}

\begin{definition}
  \label{def:length_of_resl}
  \uses{def:projv_resl}
  The length of a projective
  resolution $P_\cdot$ is 
  the highest $i$ such that $P^i$
  is nonzero.
  If there exists no such $i$, the
  length is infinity
\end{definition}

\begin{definition}
  \label{def:projdim}
  \uses{def:length_of_resl}
  Let $M$ be an $R$-module. 
  Then $\projdim M$ is the minimum 
  length of a projective resolution of $M$.
  It lives in the set $\mathbb{N} \cap \infty$.
\end{definition}

\begin{lemma}
  \label{lem:finite_resl_implies_finite_projdim}
  \uses{def:projdim}
  If there exists a projective resolution of
  $M$ with finite length, then 
  $\projdim M < \infty$.
\end{lemma}

\begin{proof}
  Expand definitions, use the definition of minimum.
\end{proof}

\begin{lemma}
  \label{lem:min_resl_has_min_length}
  Any resolution of $M$ over $R$ has 
  length at least that of the minimal resolution.
\end{lemma}

\begin{proof}
  Expand definitions, use definition of minimum
\end{proof}

\begin{proposition}
  \label{prop:existence_min_free_resl}
  \uses{def:free_resl,def:projdim}
  There exists a free resolution $F_\cdot$
  of $M$ such that length of $F_\cdot$
  is equal to the projective dimension of $M$.
\end{proposition}

\begin{proof}
  One has the generators-relations explicit construction,
  which should eventually get its own definition.
  Then one has to show that this is indeed minimal among
  projective resolutions, which I don't remember how to do
  the top of my head.
\end{proof}

%\begin{lemma}
%  \label{lem:projdim_len_minimal_free}
%  Let $R$ be a local noetherian ring, 
%  and $M$ be an $R$-module.
%  Then $\projdim M$ is equal to the length
%  of a minimal free resolution of $M$.
%\end{lemma}

\begin{definition}
  \label{def:globdim}
  \uses{def:projdim}
  The global dimension of a ring $R$
  is the supremum over all $R$-modules
  of $\projdim M$.
\end{definition}

% MARK - localization and resolutions

\begin{lemma}
  \label{lem:loczn_exact}
  Localization by a multiplicative set is an exact functor.
\end{lemma}

\begin{lemma}
  \label{lem:loczn_of_resl_is_resl}
  \uses{lem:loczn_exact}
  Let $M$ be an $R$-module.
  Let $S$ be a multiplicative set.
  Let $\ldots \to A_2 \to A_1 \to A_0 \to M \to 0$
  be a free resoltion of $M$.
  Then
  $\ldots S^{-1}(A_2) \to S^{-1}(A_1) \to S^{-1}(A_0) \to S^{-1}M \to 0$
  is a free resolution of $S^{-1}M$.
\end{lemma}


% TODO: figure out what we "really want to prove" with localizations and resolutions
%         this is kind of a cop-out

\subsection{Prime Avoidance and Variants}

Before we get too far, we present the prime avoidance lemma, which will be used
repeatedly during the rest of the blueprint.

\begin{lemma}
    [Prime Avoidance]
    \label{lem:prime_avoidance}
    (See BH 1.2.2)
    Let \(R\) be a ring, and \(\mathfrak{p}_{1}, \ldots, \mathfrak{p}_{m}\)
    be prime ideals.
    Let \(M\) be an \(R\)-module, and let \(x_{1}, \ldots, x_{n} \in M\).
    Let \(N\) be the submodule generated by \(x_{1}, \ldots, x_{n}\).

    If \(N_{\mathfrak{p}_{j}} \not\subset \mathfrak{p}_{j}M_{\mathfrak{p}_{j}}\) for
    all \(j \in \{1, \ldots, m\}\), then there exist
    \(a_{2}, \ldots, a_{n} \in R\) such that
    \(x_{1} + \sum_{i=2}^{n} a_{i}x_{i} \not\in \mathfrak{p}_{j} M_{\mathfrak{p}_{j}} \)
    for all \(j \in \{1, \ldots, m\}\).
\end{lemma}

\begin{proof}
    See BH 1.2.2
\end{proof}

\begin{corollary}
    \label{cor:prime_avoidance_ring}
    Let \(R\) be a ring, and \(\mathfrak{p}_{1}, \ldots, \mathfrak{p}_{m}\)
    be prime ideals. 
    Let \(I\) be a finitely generated ideal of \(R\).
    Then if \(I \not\subset \mathfrak{p}_{j}\) for all
    \(j \in \{1, \ldots, m\}\),
    then there exists an \(x \in I\)
    such that \(x \not\in \mathfrak{p}_{j}\) for all \(j\).
\end{corollary}

\begin{proof}
    \uses{lem:prime_avoidance}
    Take \(M = R\) in the previous theorem.
    Using the correspondence theorem for localizations, 
    the condition \(I_{\mathfrak{p}_{j}} \not\subset \mathfrak{p}_{j}R_{\mathfrak{p}_{j}}\)
    is the same as \(I \not\subset \mathfrak{p}_{j}\).
\end{proof}

We also have a few variants (see wikipedia page for "Prime Avoidance")

\begin{lemma}
    Let \(E\) be an additive subgroup of \(R\) which is multiplicatively closed.
    Let \(I_{1}, \ldots, I_{n}\) be ideals such that \(I_{3}, \ldots, I_{n}\) are prime.
    Then if \(E\) is not contained in any one of the \(I_{j}\), then \(E\) is not
    contained in their union.
\end{lemma}

\begin{lemma}
    Let \(R\) be a ring and \(\mathfrak{p}_{1}, \ldots, \mathfrak{p}_{m}\)
    be prime ideals.
    Let \(x\) be some element of \(R\) and let \(J\) be an ideal.
    Define \(I := (x) + J\).
    If \(I \not\subset \mathfrak{p}_{j}\) for all \(j\),
    then
    there exists some \(y \in J\) such that
    \(x + y \not\in \mathfrak{p}_{j}\) for all \(j\)
\end{lemma}
\begin{lemma}
    Let \(\mathfrak{p}_{1}, \ldots, \mathfrak{p}_{n}\) be prime ideals.
    Let \(I\) be an ideal contained in in their union.
    Then \(I \subset \mathfrak{p}_{i}\) for some \(i\).
\end{lemma}

\begin{proof}
    This is a weaker statement then both of the two above it.
\end{proof}

\begin{lemma}
    Let \(I_{1}, \ldots, I_{n}\) be ideals, let \(\mathfrak{p}\) be a
    prime ideal containing the intersection of them all.
    Then \(I_{i} \subset \mathfrak{p}\) for some \(i\).
    If \(\mathfrak{p}\) is exactly the intersection, then
    \(I_{i} = \mathfrak{p}\) for some \(i\).
\end{lemma}

\subsection{Dimension Theory}

A dimension theory normally takes the form of 
a theorem which says that two notions of dimension
are the same. 

Following the Appendix of Bruns-Herzog, we will use two 
notions, one related to the
\textit{lengths a systems of parameters} and one 
related to the \textit{heights of prime ideals}. 
We have not defined these terms, so we will do so first, and then
prove out theorem.

As our main tool, we use Krull's Principal Ideal Theorem
(aka the Hauptidealsatz). 
One can also use Hilbert Polynomials, see Atiyah-MacDonald.


Now, we can start to define dimension.

\begin{definition}
  \label{def:Ideal.IsPrime.height}
  \leanok
  Let $R$ be a ring, and $\mathfrak{p}$
  be a prime ideal in $R$.-
  Then the height of $\mathfrak{p}$ is the sup of
  the lengths of chains
  \[
    \mathfrak{p} = \mathfrak{p}_0 \supset \mathfrak{p}_1 \supset \dots \supset \mathfrak{p}_n
   . \]
\end{definition}


\newcommand{\height}{\operatorname{height}}


\begin{definition}
    \label{def:height_arbitrary_ideal}
    \uses{def:Ideal.IsPrime.height}
    The height of an ideal \(I \subset R\)
    is the infimum of the heights of
    prime ideals containing \(I\).
\end{definition}
    
Sanity lemma: this definition of height agrees
with the previous when \(I\) is prime (all by def).

\begin{definition}
    \label{def:krull_dim_ring}
    \uses{def:Ideal.IsPrime.height}
    The Krull dimension of the \(R\)
    is the supremum of heights of prime
    ideals \(\mathfrak{p} \subset R\).
    %In particular, \(\dim R = \height (0)\).
    % the above is not true
\end{definition}
    
The following is a nice sanity check lemma.
    
\begin{lemma}
    \label{lem:dim_local_eq_height_prime}
    \uses{def:Ideal.IsPrime.height,def:krull_dim_ring}
    Let \(R\) a ring and \(\mathfrak{p}\) a
    prime ideal, then
    \[
    \dim R_{\mathfrak{p}} = \height \mathfrak{p}
    \]
\end{lemma}
    
\begin{proof}
    This is a ``straightforward'' application of the
    correspondance theorem for localizations.
\end{proof}
    
\begin{comment}
\begin{definition}
    \label{def:primary_ideal}
    An ideal \(I\) is \(\mathfrak{p}\)-primary if it's
    radical is equal to \(\mathfrak{p}\).
\end{definition}

\begin{definition}
    \label{def:min_len_gen_set_maximal}
    \uses{def:primary_ideal}
    Let \((R,\mathfrak{m},k)\) be a local (noetherian?) ring.
    Let \(\nu(R)\)
    be the minimum of the lengths of generating sets of  
    \(\mathfrak{m}\)-primary ideals.
\end{definition}

\end{comment}

Now we state Krull's height theorem and its (partial) converse.



\begin{theorem}
    \label{thm:krull_principal_ideal}
    \uses{cor:local_maximal_gereating_set_basis_residue,def:Ideal.IsPrime.height}
    Let \(R\) be a noetherian ring, 
    and \(I\) a prinicpal ideal.
    Then each minimal prime ideal over \(I\) 
    (i.e. minimal among primes containing I)
    has height at most 1.
\end{theorem}

\begin{proof}
    See the Wikipedia page about Krull's principal ideal theorem
    (recieved June 27 2023) for a detailed proof,
    it uses passing to the quotient and the definition of
    Artinian rings.
\end{proof}

\begin{theorem}
    \label{thm:krull_height}
    \uses{thm:krull_principal_ideal}
    Let \(I = (x_{1}, \ldots, x_{n})\) be an ideal. 
    Then each minimal prime over \(I\) has height
    at most \(n\).
\end{theorem}

\begin{proof}
    Use Krull's prinipal ideal and induct on the 
    number of elements, see Wikipedia for details.
\end{proof}

\begin{theorem}
    \label{thm:genset_of_any_height_of_noetherian_height_n}
    \uses{def:height_arbitrary_ideal,cor:prime_avoidance_ring}
    Let \(R\) a noetherian ring, and \(I\) a proper 
    ideal of height \(n\).
    Then for all \(1 \leq i \leq n\),
    there exist elements \(x_{1}, \ldots, x_{i}\)
    such that
    \(\height (x_{1}, \ldots, x_{i}) = i\).
\end{theorem}

\begin{proof}
    \uses{thm:krull_height}
    We will successively apply prime avoidance.

    Base case: 
    \(I\) is not contained in any minimal prime of \(R\),
    otherwise \(I\) would have height zero (if I had height
    zero, we're already done).
    By prime avoidance, we can pick an element \(x_{1}\) which is
    not in any minimal prime. 
    Thus, any prime containing \(x_{1}\) must also contain a minimal prime, so
    all such primes have height at least 1.
    
    Pick a prime which is minimal among those containing \((x_{1})\)
    (equivalently, pick one which is minimal in the ring \(R / (x_{1})\)).
    By the above discussion, it must have height at least one. 
    By Krull's principal ideal/height theorem, it must have at most height 1,
    and we conclude it has height 1.

    Inductive step:
    Copy-paste the previous discussion, with "zero" replaced by "\(n-1\)",
    "1" replaced by "\(n\)", and "minimal prime of \(R\) " replaced by
    "minimal prime containing \((x_{1}, \ldots, x_{i-1})\)", 
    "\(x_{1}\)" replaced by "\(x_{i}\)", and
    "\((x_{1})\) replaced by \((x_{1}, \ldots, x_{i})\)".
   
\end{proof}


\begin{theorem}
    [Dimension Theorem]
    \label{thm:dimension_theorem}
    \uses{lem:dim_local_eq_height_prime,thm:krull_height,thm:genset_of_any_height_of_noetherian_height_n}
    Let \((R,\mathfrak{m},k)\) be a local noetherian ring.
    Let \(n \in \mathbb{N}\).
    Then the following are equivalent
    \begin{enumerate}[(a)]
        \item \(\dim R = n\)
        \item \(\height \mathfrak{m} = n\)
        \item \(\nu(R) = n\).
    \end{enumerate}
\end{theorem}

\begin{proof}
    The equivalence of (a) and (b) is
    Lemma \ref{lem:dim_local_eq_height_prime}
    applied to the situation of a local ring.
    The directions of (b) \(\iff\) (c) are given
    by Theorems \ref{thm:krull_height} and
    \ref{thm:genset_of_any_height_of_noetherian_height_n},
    respectively.
\end{proof}


%\begin{definition}
%  \label{def:dim_module}
%  The krull dimension of a module 
%  is the supremum of the lengths
%  of chains of prime submodules.
%\end{definition}
%
%\begin{definition}
%  \label{def:dim_ring}
%  Krull dim of ring (prime ideals
%  are prime submodules, might need 
%  to show this)
%\end{definition}

\newcommand{\Assoc}{\operatorname{Assoc}}
\newcommand{\Ann}{\operatorname{Ann}}

\begin{theorem}
    \label{thm:quotient_regular_sub_dim_1}
    Let \(R\) be a ring, and \(M\) an \(R\)-module.
    Let \(x\) be an \(M\)-regular element.
    Then \(\dim M / xM \leq \dim M - 1\).
\end{theorem}

\begin{proof}
    \uses{def:krull_dim_ring}
    As \(x\) is \(M\)-regular, we have that
    \(x \notin \mathfrak{p}\) for all
    \(p \in  \Assoc M\),
    otherwise there would be \(M\)-torsion.
    
    By definition, \(\dim M = \dim R / \Ann M\).
    The associated primes of \(M\) are the same
    as the minimal primes in \(R / \Ann M\),
    so \(x\) is not contained in any such prime.
    But then, any chain of prime ideals in 
    \((R / \Ann M) / (x)\) lifts to a chain in
    \(R / \Ann M\) of ideals which contain \((x)\).
    But then the smallest ideal in the chain 
    is not minimal since it contains \(x\), so
    we can extend it to a minimal prime.
\end{proof}

\begin{corollary}
  \label{cor:quotient_non_minimal_sub_dim_1}
  Let $x \in R$ be an element which is not in any 
  minimal prime ideal.
  Then $\dim R / (x) \leq \dim R - 1$
\end{corollary}

\begin{proof}
  \uses{thm:quotient_regular_sub_dim_1}

  Take $M = R$ in the previous theorem.
\end{proof}


Other dimension theory necessary? Probably?

\subsection{Regular elements and sequences}

\begin{definition}
  \label{def:reg_elt}
  Let $M$ be an $R$-module.
  An element $x \in R$ is $M$-regular if it does 
  not annihilate any element in $M$.
  In colon notation, it is non a member of the ideal
  $( 0 : M )$.
  If $x$ is an $R$-regular element, we simply say it is 
  a regular element.
\end{definition}

\begin{definition}
  \label{def:weak_reg_seq}
  \uses{def:reg_elt}
  Let $M$ be an $R$-module.
  Then a weak $M$-regular sequence
  is a sequence $x_1, \ldots, x_n$ 
  with $x_i \in R$ for all $i$,
  such that 
  $x_i$ is a
  $M / (x_1, \ldots, x_{i-1}) M$-regular
  element for all $i$.
  If $M = R$, we say that $x_1, \ldots, x_n$
  is a weak regular sequence. 
\end{definition}

\begin{definition}
  \label{def:reg_seq}
  \uses{def:weak_reg_seq}
  Let $M$ be an $R$-module.
  An $M$-regular sequence is a weak $M$-regular sequence
  such that $M / (x_1, \ldots, x_n) M \neq 0$.
  If $M = R$, we simply call this a regular sequence.
\end{definition}


\begin{definition}
  \label{def:reg_loc}
  A local ring $R$ is regular if 
  $\mathfrak{m}$ is generated by a system of parameters.
\end{definition}

The following is a nice sanity check lemma.

\begin{lemma}
    \label{lem:reg_maximal_dim_generators}
    \(R\) is regular if and only if 
    the minimal number of generators of its maximal
    ideal is equal to the (Krull) dimension of \(R\)
\end{lemma}

\begin{proof}
    \uses{thm:dimension_theorem}
    This follows by unwinding the definition 
    and using the dimension theorem 
    (the length of a system of paramaters is \(\nu(R)\)
    by definition).
\end{proof}

\begin{comment}
\begin{definition}
  \label{def:reg_ring}
  \uses{def:reg_loc}
  A ring $R$ is regular if 
  $R_{\mathfrak{p}}$ is regular
  for every $\mathfrak{p} \in R$.
\end{definition}

\begin{proposition}
  \label{prop:reg_def_equiv}
  \uses{def:reg_loc}
  The following are equivalent
  Firstly, $R$ is regular.
  Secondly, the zariski cotangent space
  is a vector space of dimension $\dim R$.
\end{proposition}
\end{comment}

\begin{proof}
  See in \verb|adjunction_blob.txt|
\end{proof}

\begin{lemma}
  \label{lem:reg_int_dom}
  \uses{def:reg_loc}
  Every regular ring is an integral domain.
\end{lemma}

\begin{proof}
  \uses{cor:local_maximal_gereating_set_basis_residue, thm:krull_height}

  See BH 2.2.3

\end{proof}

\begin{proposition}[BH 2.2.4]
  \label{prop:reg_quot_sys_param}
  Let $R$ be a regular local ring. 
  Then $R / I$ is regular local if and only if 
  $I$ is generated by a (regular) system of parameters
  (I.e. a generating set for $\mathfrak{m}$).
\end{proposition}

\begin{proof}
  \uses{lem:reg_int_dom}
  This proof uses the following facts:

  * a Nakayama corollary
  
  * the fact that regular rings are integral domains

  * the fact that you can't have a proper containment
    of integral domains with the same dimension
\end{proof}

\begin{proposition}
  \label{prop:reg_loc_maximal_reg_seq}
  \uses{def:reg_loc,def:reg_seq}
  A local ring $R$ is regular if and only if
  its maximal ideal is generated by a regular sequence.
\end{proposition}

\begin{proof}
  \uses{lem:reg_int_dom,
    thm:reg_seq_part_of_sys_param,
    lem:reg_maximal_dim_generators,
    prop:reg_quot_sys_param
  }
  See Bruns-Herzog 2.2.5. This uses BH 2.2.4 
  (which is Proposition \ref{prop:reg_quot_sys_param}) for the forward
  direction,
  and for the reverse direction, we use BH 1.2.12 
  (aka Theorem \ref{thm:reg_seq_part_of_sys_param}) 
  and the fact that
  the minimal number of generators for $\mathfrak{m}$ is at least
  $\dim R$, which is Lemma \ref{lem:reg_maximal_dim_generators}.
\end{proof}

\begin{comment}
\subsection{Support Theory}

\begin{definition}
  \label{def:annihilator}
  The annihilator of a module M is the ideal
  of elements $r \in R$ such that $rm = 0$ for 
  all $m \in M$.
\end{definition}

\begin{definition}
  \label{def:support_module}
  The support of a module is the set of prime ideals
  $\mathfrak{p} \ins R$ such that 
  $M_{\mathfrak{p}} \neq 0$.
\end{definition}

\begin{lemma}
  \label{lem:annihilator_zero_locus_support}
  \uses{def:annihilator,def:support_module}
  The zero locus of the annihilator of a module
  is the same as the support of that module.
  I.e. the set of prime ideals containing the 
  $\Ann_R(M)$ is the support of $M$.
\end{lemma}

\begin{proof}
  omitted for now
\end{proof}

\subsection{Selected Results on Primary Decomposition}

\begin{definition}
    \label{def:primary}
    Given a prime ideal \(\mathfrak{p}\) of the ring \(R\),
    an ideal \(I\) is \(\mathfrak{p}\)-primary if
    for all \(x, y \in I\), either \(x \in I\) or \(y^n \in I\)
    for some \(n\).
    (This is already in mathlib)
\end{definition}

\begin{lemma}
    \label{lem:radical_isPrime_of_isPrimary}
    \uses{def:primary}
    If \(\mathfrak{q}\) is a primary ideal, then
    \(\sqrt{q}\) is a prime ideal.
\end{lemma}

\begin{proof}

\end{proof}

\begin{lemma}
    \label{lem:isPrimary_inf_finset}
    \uses{def:primary}
    Let \(I_{1}, \ldots, I_{n}\) be \(\mathfrak{p}\)-primary ideals.
    Then their intersection is \(\mathfrak{p}\)-primary.
\end{lemma}

\begin{proof}
    ``isPrimary_inf'' in mathlib is this but for only two ideals,
    so just use induction.
\end{proof}

\begin{definition}
    \label{def:primary_decomp}
    \uses{def:primary}
    A primary decomposition of an ideal \(I \subset R\) is
    an expression of \(I\) as an intersection of primary ideals,
    that is a collection of ideals
    \(\mathfrak{q}_{i}\), \(1 \leq i \leq n\),
    such that
    \[
    I = \bigcap_{i=1}^{n} \mathfrak{q}_{i} 
    .\]
\end{definition}

\begin{definition}
    \label{def:decomposable}
    \uses{def:primary_decomp}
    An ideal \(I\) is decomposable there exists a primary decomposition
    with intersection \(I\).
\end{definition}

\begin{definition}
    \label{def:minimal_primary_decomp}
    \uses{def:primary_decomp}
    A minimal primary decomposition is a primary decomposition such that
    \begin{enumerate}[(i)]
        \item The ideals \(\sqrt{\mathfrak{q}_{i}}\) are distinct
        \item
            \[
            \bigcap_{i \neq j}^{} \mathfrak{q}_{j} \not\subset \mathfrak{q}_{i} 
            \]
            for all \(1 \leq i \leq n\).
    \end{enumerate}
\end{definition}


\begin{lemma}
    \label{lem:has_minimal_primary_decomp_of_decomposable}
    \uses{def:decomposable}
    Any decomposable ideal has a minimal primary decomposition
\end{lemma}

\begin{proof}
    \uses{lem:isPrimary_inf_finset}
    Reduce to the case where there is a single prime ideal per
    primary ideal by lemma \ref{lem:isPrimary_inf_finset},
    and then if there are any superflous terms breaking condition
    (ii), remove them.
\end{proof}


\begin{theorem}
    [First Uniqueness Theorem for Primary Decomposition]
    \label{thm:first_uniqueness_primary_decomposition}
    \uses{lem:has_minimal_primary_decomp_of_decomposable, lem:prime_avoidance}

\end{theorem}

\begin{proof}
    See Atiyah-MacDonald, Theorem 4.5
\end{proof}



\subsection{Associated Primes}


\begin{proposition}
  \label{prop:assoc_primes_def_equiv}
  Let $R$ a ring, and $M$ an $R$-module.
  Let $\mathfrak{p}$ be a prime ideal, 
  then 
  the following are equivalent:
  \begin{enumerate}[label=(\roman*)]
  \item $\mathfrak{p} = \Ann_R (m)$ for some element $m \in M$.
  \item $R / \mathfrak{p}$ embeds into $M$.
      % \item something about prime submodules
  \end{enumerate}
\end{proposition}

\begin{definition}
  \label{def:assoc_primes}
  \uses{prop:assoc_primes_def_equiv}
  The set $\Assoc_R (M)$ is the set of
  primes satisfying the preivious proposition
\end{definition}

\begin{proposition}
  \label{prop:assoc_primes_in_support}
  \uses{lem:annihilator_zero_locus_support}
  The associated primes of $M$ are in the
  support of $M$.
\end{proposition}

\begin{lemma}
  \label{lem:assoc_iff_minimal_in_support}
  \uses{prop:assoc_primes_in_support}
  The assocated primes of $M$ are precisely
  the primes which are minimal in the support of $M$.
\end{lemma}

% Do we need other things about associated primes?
% relation to primary decomposition?
% associated primes are minimal?

\begin{lemma}
  \label{lem:assoc_prime_zero_divisor}
  \uses{thm:first_uniqueness_primary_decomposition}
  An element $x \in R$ is a zero-divisor iff
  it is in an associated prime.
  (geometrically, it vanishes at an associated point)
\end{lemma}

\end{comment}



\subsection{Depth and Regular Sequences}

This mostly follows Bruns-Herzog 1.2.
We are building up to Bruns-Herzog 1.2.12.

Actually maybe we don't need depth??

\begin{definition}
  \label{def:depth}
	\uses{def:weak_reg_seq}
  The depth of a module $M$ is . . . 
\end{definition}

\begin{theorem}[BH 1.2.12]
  \label{thm:reg_seq_part_of_sys_param}
  \uses{def:reg_seq,cor:quotient_non_minimal_sub_dim_1,lem:assoc_prime_zero_divisor}
  Any regular sequence is part of a system of parameters
\end{theorem}

\begin{proof}
  %\uses{def:reg_seq,cor:quotient_non_minimal_sub_dim_1,lem:assoc_prime_zero_divisor}
  \uses{thm:dimension_theorem,
      lem:assoc_prime_zero_divisor,
      lem:prime_avoidance,
      lem:assoc_iff_minimal_in_support}

    This is 
    from BH chapter 1, and uses associated primes.
    It also uses the definition of dimension of a module,
    
    Let \(x \in \mathfrak{m}\) an \(M\)-regular element.
    As \(x \in R\) is \(M\)-regular,  
    it is not in any associated prime (Lemma \ref{lem:assoc_prime_zero_divisor}).
    
    Then, as the associated primes are minimal among the primes
    that contain the annihilator (which is the support),
    we have that the previous statement implies that
    \[
    \dim M / xM \leq \dim M - 1
    \] 
    by corollary \ref{cor:quotient_non_minimal_sub_dim_1}
    
    By dimension theory (see BH A.4), this inequality is an equality.
    
    Given a regular sequence, iterate this process. 
    %sketchy:
    Once we get to the end of this process, we can extend this sequence
    (by prime avoidance?)
    to a sequence of
    %end sketchy
    % I think the sketchy part, even if it's true, is not the most efficient way of doing this.
    \(\dim M\) elements, by definition (of system of parameters) we have a system of
    parameters for M.

\end{proof}


\begin{comment}
\subsection{Auslander-Buchsbaum Formula}

\begin{lemma}
  \label{lem:module_assoc_maximal_tensor_free_summand}
  BH 1.3.4
\end{lemma}

\begin{proof}
  \uses{prop:assoc_primes_def_equiv}
  Uses

  * def of associated primes

  * a fact about associated primes giving and embedding

  * some facts about commutative squares (maybe already in lean?)

  * tensor products of modules

  * Nakayama (a corollary, like Atiyah-MacDonald 2.8 
    but uses maps, might just follow from AM 2.8)
\end{proof}

\begin{lemma}
  \label{lem:projdim_preserved_quotient_reg_elt}
  BH 1.3.5
\end{lemma}

\begin{proof}
  This uses 

  * BH 1.1.5

  * The "tor" characterization of projective dimension

\end{proof}

\begin{theorem}
  \label{thm:auslander_buchsbaum_formula_old}
  depth + projdim = depth
\end{theorem}

\begin{proof}
  \uses{
    lem:module_assoc_maximal_tensor_free_summand,
    lem:projdim_preserved_quotient_reg_elt,
  }
\end{proof}

\end{comment}


%TODO: figure out if we will actually use this.

%\subsection{Cohen-Macaulay Rings}
%
%
%Let $(R,\mathfrak{m},k)$ be a noetherian local ring.
%
%\begin{definition}
%  \label{def:local_cm}
%  \uses{def:depth}
%  An $R$-module $M$ is 
%  Cohen-Macaulay if $\depth M = \dim M$.
%  We say that $R$ is a Cohen-Macaulay ring if
%  $R$ is Cohen-Macaulay as a module.
%\end{definition}
%
%\begin{definition}
%  \label{def:maxml_local_cm}
%  \uses{def:dim_module}
%  $M$ is maximal Cohen-Macaulay if 
%  it has $\dim M = \dim R$.
%\end{definition}
%
%Now, let $R$ be an arbitrary noetherian ring.
%
%\begin{definition}
%  \label{def:cm}
%  \uses{def:support_module,def:local_cm}
%  $M$ is Cohen-Macaulay if 
%  $M_\mathfrak{m}$ is for all maximal
%  ideals $\mathfrak{m}$ in the support of $M$.
%\end{definition}
%
%\begin{corollary}
%  \label{cor:zero_is_cm}
%  \uses{def:cm}
%  The zero module is Cohen-Macaulay.
%\end{corollary}
%
%\begin{definition}
%  \label{def:maxml_cm}
%  \uses{def:maxml_local_cm}
%  $M$ is maximal Cohen-Macaulay if 
%  if $M_\mathfrak{m}$ is for all 
%  maximal ideals $\mathfrak{m}$.
%  I.e. there is no support condition as before.
%\end{definition}
%
%\begin{theorem}
%  [BH 2.1.2]
%  \label{thm:cm_reg_seq_iff_sys_param}
%  Let $M$ be a Cohen-Macaulay $R$-module.
%  Then $x_1, \ldots, x_n$ is a regular sequence
%  if and only if it is part of a system of parameters.
%\end{theorem}
%
%\begin{proof}
%  \uses{
%    prop:assoc_primes_in_support,
%  }
%  Uses
%  
%  * BH 1.2.13
%
%  * the fact that the associated primes are in the support
%
%  * some stuff about depth and grade and dimension
%
%  * 1.6.19
%
%  OR, use hochster's proof in
%  http://www.math.lsa.umich.edu/~hochster/615W14/CM.pdf
%\end{proof}
%
%\begin{proposition}
%  \label{prop:cm_syzygy}
%  Let $R$ be Cohen-Macaulay.
%  Then any dth syzygy module is a maximal cohen-macaulay or zero
%
%\end{proposition}
%
%\begin{proof}
%  https://math.stackexchange.com/questions/159390/bruns-herzog-problem-2-1-26-page-64
%\end{proof}

\begin{comment}
\subsection{Proof of Auslander-Buchbaum-Serre}



\begin{lemma}
  \label{lem:reg_implies_finite_global}
  \uses{def:globdim}
  Let $R$ be a regular local ring.
  Then $R$ has finite global dimension.
  That is, any finitely generated module $R$ has finite
  projective dimension.
\end{lemma}

\begin{proof}
  %TODO
\end{proof}

\begin{theorem}
  [Ferrand-Vasconcelos, BH 2.2.8]
  \label{thm:ferrand_vasconcelos}
  \uses{def:reg_seq,def:projdim}

  Let $(R,\mathfrak{m},k)$ be a local noetherian ring.
  Let $I$ be a nonzero ideal with finite projective dimension.
  If $I / I^2$ is a free $R$-module, 
  then $I$ is generated by a regular sequence.
\end{theorem}

\begin{proof}
  Since $I$ has finite projective dimension, it has a finite free
  resolution.
  Thus, by 1.4.6 it has must have an $R$-regular element $x$.
  . . . finish the proof . . . 
\end{proof}

% MARK: Auslander-Buchsbaum-Serre (the goal)

\begin{theorem}[Auslander-Buchsbaum-Serre Criterion, BH 2.2.7]
  \label{thm:auslander_buchsbaum_serre}
  Let $(R,\mathfrak{m},k$ be a noetherian local ring.
  The following are equivalent:
  \begin{enumerate}[label=(\roman*)]
    \item $R$ is regular.
    \item $R$ has finite global dimension.
    \item $\projdim k < \infty$
  \end{enumerate}
\end{theorem}

\begin{proof}
  \uses{
    lem:reg_implies_finite_global,
    %thm:ferrand_vasconcelos,
    %prop:reg_loc_maximal_reg_seq
	  %thm:loc_ring_reg_iff_fin_globdim,
	  %cor:koszul_min_resl_residue_if_reg
  }

  (i) $\implies$ (ii) is precisely Lemma \ref{lem:reg_implies_finite_global}

  (ii) $\implies$ (iii) follows by applying the definition of global dimension with $M=k$

  (iii) $\implies$ (i) is a special case of Theorem \ref{thm:ferrand_vasconcelos},
  using \ref{prop:reg_loc_maximal_reg_seq} to 
  conclude regularity.

  %\ref{thm:ferrand_vasconcelos}.
\end{proof}

\begin{theorem}[Regular Rings Localize, BH 2.2.9]

  \label{thm:regular_rings_localize}
  Let $R$ be a regular local ring, and let
  $\mathfrak{p}$ be a prime ideal in $R$.
  Then $R_{\mathfrak{p}}$ is a regular local ring.
\end{theorem}

% probably we want to carry around the dimension n

\begin{proof}
  \uses{thm:auslander_buchsbaum_serre,prop:existence_min_free_resl,lem:loczn_of_resl_is_resl,lem:min_resl_has_min_length}
  By Auslander-Buchsbaum-Serre, it is enough to show that 
  $R_{\mathfrak{p}} / \mathfrak{p}R_{\mathfrak{p}}$ has finite projective dimension.
  Also by Auslander-Buchsbaum-Serre, we know that 
  $k = R / \mathfrak{m}$ has finite projective dimension.
  Then $k$ has a minimal free resolution of finite length by
  Proposition \ref{existence_min_free_resl}.
  By the fact that the loclization of a resolution is a resolution,
  We get a finite resolution for 
  $R_{\mathfrak{p}} / \mathfrak{p}R_{\mathfrak{p}}$,
  thus any minimal resolution is also finite, giving us what we want.

\end{proof}
\end{comment}
