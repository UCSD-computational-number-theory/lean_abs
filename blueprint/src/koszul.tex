All rings are assumed to be unital, commutative and Noetherian without specify.

\subsection{Koszul complex}

\begin{definition}



\begin{definition}
	\label{def:minimal_free_resl}
	\uses{def:free_resl}
\end{definition}

%\begin{proposition}
	%\label{prop:min_free_resl_unique}
%\end{proposition}

\begin{proposition}
	\label{prop:loc_resl_min_iff_basis_to_gen}
	\uses{def:minimal_free_resl, lem:nakayama}
\end{proposition}


\begin{definition}[koszul_complex]
	\label{def:koszul_complex}
	\uses{def:ExteriorAlgebra}
	Let $R$ be a (Noetherian?) ring, $M$ be an (finitely generated?) $R$-module and $a \in M$.
	Then for each graded component in $\bigwedge_R M$ there is a restriction of $R$-linear map $d_a := \underline{\ } \wedge a: \bigwedge_R M \to \bigwedge_R M$:
	$$d_i: \bigwedge_R^i M \to \bigwedge_R^{n+1}M \text{ where } x \mapsto x \wedge a \text{ for all } x \in \bigwedge_{R}^i M$$
	So that $d_a = \bigoplus_{i \in \mathbb{Z}}d_i$, the \textbf{Koszul complex} is then the cochain complex
	$$\begin{tikzcd}
		0 \arrow[r] & \bigwedge^{0}_R M \arrow[r, "d_{0}"] & \bigwedge^1_R M \arrow[r, "d_1"] & \dots \bigwedge_R^i M\arrow[r, "d_i"] & \bigwedge_R^{i+1} M \arrow[r] & \dots
	\end{tikzcd}$$
	It's indeed a complex as $d_a^2 = 0$.
\end{definition}

\begin{definition}[Dual Koszul complex]
	\label{def:koszul_complex_dual}
	\uses{def:koszul_complex}
	If $M \cong R^m$ is a free $R$-module, then applying the contravariant functor $\operatorname{Hom}_R(\ , R)$ gives us an isomorphic chain complex 
	$$K_{\bullet}(a) :\begin{tikzcd}
	\bigwedge^{i+1}_R R^m \arrow[r, "d_{i+1}"] & \bigwedge^i_R R^m \arrow[r, "d_i"] & \bigwedge^{i-1}_R R^m \arrow[r] & \dots R^m \arrow[r, "d_1"] & R \arrow[r] & 0
	\end{tikzcd}$$
\end{definition}

\begin{definition}
	\label{def:complex_tensor_product}
	Let $K, L$ be two chain complexes, the tensor product of $K$ and $L$, denoted as $K \otimes L$ is defined to be the chain complex
	$$(K \otimes L)_r := \bigoplus_{p + q = r}K_p \otimes_R L_q$$
	Where the differential is defined as $d = (-1)^q d_K + d_L$.
\end{definition}

\begin{lemma}
	\label{lem:koszul_of_length_2}
	\uses{def:koszul_complex_dual}
	$K(x)$ is exact exept for the rightmost side iff $x$ is regular.
\end{lemma}

\begin{lemma}
	\label{lem:koszul_tensor_as_mapping_cone}
	\uses{lem:koszul_of_length_2, def:complex_tensor_product}
	Let $L$ be a bounded below complex of $R$-modules, and $x \in R$, then
	$$(L \otimes K(x))_i \cong L_i \oplus L_{i-1}$$
	Where $d_i: (L_i \oplus L_{i-1}) \to (L_{i-1}\oplus L_{i-2})$ is given by the matrix
	$$d_i = \begin{pmatrix}d_L & (-1)^i \cdot x \\ 0 & d_L\end{pmatrix}$$
\end{lemma}

\begin{theorem}
	\label{thm:long_koszul_build_by_tensoring}
	\uses{lem:koszul_tensor_as_mapping_cone}
	Let $(a_1, \dots, a_n) \in R^n$ then:
	$$K(a_1, \dots, a_n) \cong K(a_1) \otimes \dots \otimes K(a_n)$$
\end{theorem}

\begin{lemma}
	\label{lem:koszul_induces_long_exact_seq}
	\uses{thm:long_koszul_build_by_tensoring}
	Let $L$ be a complex of $R$-modules, then there is a long exact sequence
\[\begin{tikzcd}
	\dots & {H_i(L)} & {H_i(L)} & {H_i(L\otimes K(a))} & {H_{i-1}(L)} & \dots
	\arrow[from=1-1, to=1-2]
	\arrow["{\cdot a}", from=1-2, to=1-3]
	\arrow[from=1-3, to=1-4]
	\arrow[from=1-4, to=1-5]
	\arrow[from=1-5, to=1-6]
\end{tikzcd}\]
\end{lemma}

\begin{proof}
	$L \otimes K(a)$ is in exactly the mapping cone.
\end{proof}

\begin{corollary}
	\label{cor:tensor_koszul_induced_long_exact_seq}
	\uses{thm:long_koszul_build_by_tensoring}
	Denote $a = (a_1, \dots, a_n)$ and $a' = (a_1, \dots, a_{n+1})$ then there is a long exact sequence:
\[\begin{tikzcd}
	{H_i(K(a) \otimes M)} & {H_i(K(a) \otimes M)} & {H_i(K(a') \otimes M)} & {H_{i-1}(K(a) \otimes M)} & 
	\arrow[shorten <=8pt, from=1-1, to=1-2]
	\arrow[shorten <=8pt, "{\cdot a_{n+1}}", from=1-2, to=1-3]
	\arrow[shorten <=8pt, from=1-3, to=1-4]
	\arrow[shorten <=8pt, from=1-4, to=1-5]
\end{tikzcd}\]
\end{corollary}

\begin{theorem}
	\label{thm:reg_seq_implies_koszul_exact}
	\uses{cor:tensor_koszul_induced_long_exact_seq, def:reg_seq}
	Let $M$ be an $R$-module, and $a = (a_1, \dots, a_n)$ be a weakly $M$-regular sequence.
	Then $H_i(K(a) \otimes M) = 0$ for all $i > 0$.
\end{theorem}

\begin{corollary}
	\label{cor:reg_in_jrad_iff_koszul_exact}
	\uses{thm:reg_seq_implies_koszul_exact, lem:nakayama, def:reg_seq}
	If $M$ is a finitely generated $R$-module and $a = (a_1, \dots, a_n)$ be a sequence in the Jacobson radical $J$ of $R$.
	If $H_1(K(a) \otimes M) = 0$, then $a$ is $M$-regular.
	In fact, this does not depends on the order.
\end{theorem}

\begin{lemma}
	\label{thm:koszul_complex_res_quotient_ring}
	\uses{def:free_resl}
\end{lemma}

\begin{lemma}
	\label{lem:tor_measures_koszul_homology}
	\uses{thm:koszul_complex_res_quotient_ring}
	$$H_i(K(a) \otimes M) \cong \tor_i^R (R/(a), M)$$
\end{lemma}

\begin{lemma}
	\label{lem:koszul_homology_mesures_depth}
	\uses{def:depth}
	Let $a = (a_1, \dots, a_n)$ and $I = \la a_1, \dots, a_n\ra$ then
	$$\depth_I(M) = n - \max\{i \mid H_i(K(a) \otimes M) \neq 0\}$$
\end{lemma}

\begin{lemma}
	\label{lem:tor_eq_pd_if_local}
	\uses{def:projdim}
	Let $(R, \mathfrak{m}, k)$ be a local ring, then
	$$\tor_i^R(k, M) = \pd M$$
\end{lemma}

\begin{proposition}
	\label{prop:globdim_eq_projdim_of_residue}
	\uses{lem:fin_gen_proj_over_local_is_free, lem:tor_eq_pd_if_local, def:globdim}
\end{proposition}

\begin{lemma}
	\label{lem:projdim_ker_eq_one_less}
	Let $M$ be an $R$-module such that $\pd M < \infty$, let $P$ be a projective module such that $\varphi: P \to M$ is surjective.
	Then let $K = \ker \varphi$ then either $K$ and $M$ are projecive or
	$$\pd K = \pd M - 1$$
\end{lemma}

\begin{lemma}
	\label{lem:koszul_homotopy}
	\uses{def:koszul_complex_dual}
\end{lemma}

\begin{proposition}
	\label{prop:koszul_homology_annil_by_maximal}
	\uses{lem:koszul_homotopy}
\end{proposition}

\begin{theorem} (Auslander-Buchsbaum formula) Let $(R, \mathfrak{m})$ be a local ring, and $M$ is a finitely generated $R$-module, then:
	\label{thm:auslander_buchsbaum_formula}
	\uses{lem:tor_eq_pd_if_local, 
	lem:koszul_homology_mesures_depth,
	lem:tor_eq_pd_if_local,
	lem:tor_measures_koszul_homology,
	lem:projdim_ker_eq_one_less,
	prop:koszul_homology_annil_by_maximal,
	prop:loc_resl_min_iff_basis_to_gen}
	$$\pd M = \depth R - \depth M$$
\end{theorem}

\begin{lemma}
	\label{lem:depth_leq_ht}
	\uses{thm:auslander_buchsbaum_formula, thm:krull_height}
	Let $R$ be a Noetherian ring and an ideal $I$ conains a regular sequence $a_1, \dots, a_n$, then $\height I \geq n$.
\end{lemma}

\begin{corollary}
	\label{lem:depth_eq_ht_if_gen}
	\uses{lem:depth_leq_ht}
	Let $R$ be a Noetherian local ring and an ideal $I$ is generated by a regular sequence, then $\height I = \depth_I A = n$.
\end{corollary}

\begin{corollary}
	\label{cor:koszul_min_resl_residue_iff_reg}
	\uses{lem:tor_measures_koszul_homology, cor:reg_in_jrad_iff_koszul_exact, prop:loc_resl_min_iff_basis_to_gen}
	If $a_1, \dots, a_n$ is a regular sequence in $(R, \mathfrak{m})$, then the Koszul complex $K(a_1, \dots, a_n)$ is a minimal free resolution of $k = R/(a_1, \dots, a_n)$.
\end{corollary}


%\begin{theorem}
	%\label{thm:reg_loc_iff_max_gen_by_reg_seq}
	%\uses{def:reg_loc, def:reg_seq, thm:krull_height}
	%A local ring is regular iff its maximal ideal is generated by a regular sequence.
%\end{theorem}



\begin{theorem} (Auslander-Buchsbaums-Serre)
	\label{thm:loc_ring_reg_iff_fin_globdim}
	\uses{lem:depth_eq_ht_if_gen,
	cor:koszul_min_resl_residue_iff_reg,
	prop:reg_loc_maximal_reg_seq,
	prop:globdim_eq_projdim_of_residue}
	A local ring $R$ has finite global dimension iff it's regular.
\end{theorem}
